\chapter{BKNR Indices}

\section{ Introduction}

In the framework we built as a backend for the `eboy.com' website,
we built a prevalence layer that could handle CLOS objects. These
CLOS objects all had an `ID', and could be indexed over other
slots as well. For example, we heavily used ``keyword indices'' that
could give back all objects that had a certain keyword stored in a
slot. However, the slot indices were built into a very big
`define-persistent-class' macro, and could not easily be extended
or used on their own.

This index layer is now built using the metaobject protocol, and
has a CLOS method protocol to access indices, so that new index
classes can easily be added.

This tutorial will show you how to create CLOS classes with slot
indices, class indices, and how to create custom indices and use
them with your classes.


\section{ Obtaining and loading BKNR slot indices}

You can obtain the current CVS sources of BKNR by following the
instructions at `http://bknr.net/blog/bknr-devel'. Add the `src'
directory of BKNR to your `asdf:*central-registry*', and load the
indices module by evaluating the following form:

\begin{Verbatim}[fontsize=\small,frame=leftline,framerule=0.9mm,rulecolor=\color{gray},framesep=5.1mm,xleftmargin=5mm,fontfamily=cmtt]
(asdf:oos 'asdf:load-op :bknr-indices)
\end{Verbatim}
Then switch to the `bknr.indices' package to try out the tutorial.

\begin{Verbatim}[fontsize=\small,frame=leftline,framerule=0.9mm,rulecolor=\color{gray},framesep=5.1mm,xleftmargin=5mm,fontfamily=cmtt]
(in-package :bknr.indices)
\end{Verbatim}


\section{ A simple indexed class}



\subsection{ A standard non-indexed class}

We begin by defining a simple class called GORILLA. Gorillas have
a name, and a description keyword.

\begin{Verbatim}[fontsize=\small,frame=leftline,framerule=0.9mm,rulecolor=\color{gray},framesep=5.1mm,xleftmargin=5mm,fontfamily=cmtt]
(defclass gorilla ()
  ((name        :initarg :name
      :reader gorilla-name
      :type string)
   (description :initarg :description
      :reader gorilla-description)))

(defmethod print-object ((gorilla gorilla) stream)
  (print-unreadable-object (gorilla stream :type t)
    (format stream "~S" (gorilla-name gorilla))))
\end{Verbatim}
We can create a few gorillas to test the class. To refer to these
gorillas later on, we have to store them in a list. We can then
write functions to search for gorillas.

\begin{Verbatim}[fontsize=\small,frame=leftline,framerule=0.9mm,rulecolor=\color{gray},framesep=5.1mm,xleftmargin=5mm,fontfamily=cmtt]
(defvar *gorillas* nil)

(setf *gorillas*
      (list
       (make-instance 'gorilla :name "Lucy"
            :description :aggressive)
       (make-instance 'gorilla :name "Robert"
            :description :playful)
       (make-instance 'gorilla :name "John"
            :description :aggressive)))

(defun all-gorillas ()
  (copy-list *gorillas*))

(defun gorilla-with-name (name)
  (find name *gorillas* :test #'string-equal
   :key #'gorilla-name))

(defun gorillas-with-description (description)
  (remove description *gorillas* :test-not #'eql :key
     #'gorilla-description))

(all-gorillas)
; => (#<GORILLA "Lucy"> #<GORILLA "Robert"> #<GORILLA "John">)
(gorilla-with-name "Lucy")
; => #<GORILLA "Lucy">
(gorillas-with-description :aggressive)
; => (#<GORILLA "Lucy"> #<GORILLA "John">)
(gorilla-with-name "Manuel")
; => NIL
\end{Verbatim}
What we would like to do however, is have the object system index
these objects for us. This is achieved by using INDEXED-CLASS as
the metaclass for the gorilla class. The `INDEXED-CLASS' has its
own slot-definition objects called `INDEX-DIRECT-SLOT-DEFINITION'
and `INDEX-EFFECTIVE-SLOT-DEFINITION'. Using these classes, we can
specify additional initargs to our slot definitions.



\subsection{ Additional slot initargs}

The following additional initargs are available:

`INDEX' - A class name that specifies the class of the index to
use. For example `SLOT-INDEX', `KEYWORD-INDEX' or
`KEYWORD-LIST-INDEX'.

`INDEX-INITARGS' - Additional arguments that are passed to
`INDEX-INITIALIZE' when creating the index.

`INDEX-READER' - A symbol under which a query function for the
index will be stored.

`INDEX-ALL' - A symbol under which a function returning all the
values of the index will be stored.

`INDEX-SUBCLASSES' - Determines if instances of subclasses of this
class will be indexed in the slot index also. Defaults to `T'.



\subsection{ A simple indexed class}

Using the `INDEXED-CLASS', we can redefine our gorilla example.

\begin{Verbatim}[fontsize=\small,frame=leftline,framerule=0.9mm,rulecolor=\color{gray},framesep=5.1mm,xleftmargin=5mm,fontfamily=cmtt]
(defclass gorilla ()
  ((name :initarg :name :reader gorilla-name
    :index-type slot-index
    :index-initargs (:test #'equal)
    :index-reader gorilla-with-name
    :index-values all-gorillas)
   (description :initarg :description
      :reader gorilla-description
      :index-type keyword-index
      :index-reader gorillas-with-description))
  (:metaclass indexed-class))
\end{Verbatim}
We have to recreate the gorillas though, as the old instances
don't get updated for now.

\begin{Verbatim}[fontsize=\small,frame=leftline,framerule=0.9mm,rulecolor=\color{gray},framesep=5.1mm,xleftmargin=5mm,fontfamily=cmtt]
(make-instance 'gorilla :name "Lucy" :description :aggressive)
(make-instance 'gorilla :name "Robert" :description :playful)
(make-instance 'gorilla :name "John" :description :aggressive)))

(all-gorillas)
; => (#<GORILLA "Lucy"> #<GORILLA "Robert"> #<GORILLA "John">)
(gorilla-with-name "Lucy")
; => #<GORILLA "Lucy">
; T
(gorillas-with-description :aggressive)
; => (#<GORILLA "John"> #<GORILLA "Lucy">)
; T
\end{Verbatim}


\subsection{ Class indices}

We can also create indices that are not bound to a single
slot. These indices are called `CLASS-INDICES'. For example, we
can add two slots for the coordinates of the gorilla, and a class
index of type `ARRAY-INDEX' that will index the two slots `X' and
`Y' of the gorilla in an array of dimensions `256x256'. Note that
redefining the class conserves the existing indices.

\begin{Verbatim}[fontsize=\small,frame=leftline,framerule=0.9mm,rulecolor=\color{gray},framesep=5.1mm,xleftmargin=5mm,fontfamily=cmtt]
(defclass gorilla ()
  ((name :initarg :name :reader gorilla-name
    :index-type slot-index
    :index-initargs (:test #'equal)
    :index-reader gorilla-with-name
    :index-values all-gorillas)
   (description :initarg :description
      :reader gorilla-description
      :index-type keyword-index
      :index-reader gorillas-with-description)
   (x :initarg :x :reader gorilla-x)
   (y :initarg :y :reader gorilla-y))
  (:metaclass indexed-class)
  (:class-indices (coords :index-type array-index
           :slots (x y)
           :index-reader gorilla-with-coords
           :index-initargs (:dimensions '(256 256)))))

(make-instance 'gorilla :name "Pete" :description
          :playful :x 5 :y 8)

(gorilla-with-coords '(5 8))
; => #<GORILLA "Pete">
(all-gorillas)
; => (#<GORILLA "Lucy"> #<GORILLA "Robert">
;     #<GORILLA "John"> #<GORILLA "Pete">)
(gorillas-with-description :playful)
; => (#<GORILLA "Pete"> #<GORILLA "Robert">)
; T

(let ((lucy (gorilla-with-name "Lucy")))
  (with-slots (x y) lucy
    (setf x 0 y 0)))

(gorilla-with-name "Lucy")
; => #<GORILLA "Lucy">
; T
(gorilla-with-coords '(0 0))
; => #<GORILLA "Lucy">
\end{Verbatim}


\section{ Creating indexed classes}

Adding indexes to a class is very simple. The class has to have
the metaclass `INDEXED-CLASS', or a class deriving from
`INDEXED-CLASS'.



\subsection{ Slot indices}

`INDEXED-CLASS' uses its own `EFFECTIVE-SLOT-DEFINITION' and
`DIRECT-SLOT-DEFINITION' which add indices to slots. A slot
definition in the `DEFCLASS' form supports additional keyword
arguments:

`:INDEX' - Specifies an existing index to use as slot-index for this slot.

`:INDEX-TYPE' - Specifies the class of the index to be used for this
slot.

`:INDEX-INITARGS' - Specifies additional initargs to be given to
`INDEX-CREATE' when creating the index. The slot-name is given as
the `:SLOT' keyword argument to `INDEX-CREATE'.

`:INDEX-READER' - Specifies the name under which a query function
for the created index will be saved.

`:INDEX-VALUES' - Specifies the name under which a function returning
all the objects stored in the created index will be saved.

`:INDEX-MAPVALUES' - Specifies the name under which a function
applying a function to all the objects stored in the created index
will be saved.

`:INDEX-SUBCLASSES' - Specifies if subclasses of the class will
also be indexing in this index. Default is `T'.

For each `DIRECT-SLOT-DEFINITION' of an indexed class with the
`:INDEX' keyword, an index is created and stored in the
`DIRECT-SLOT-DEFINITION'. All the direct indexes are then stored
in the `EFFECTIVE-SLOT-DEFINITION' (indexes with
`INDEX-SUBCLASSES = NIL' will not).

Every access to the slot will update the indices stored in the
`EFFECTIVE-SLOT-DEFINITION'. When the slot is changed, the object
is removed from all the slot indices, and added after the slot
value has been changed. When a slot is made unbound, the object is
removed from the slot indices.

\begin{Verbatim}[fontsize=\small,frame=leftline,framerule=0.9mm,rulecolor=\color{gray},framesep=5.1mm,xleftmargin=5mm,fontfamily=cmtt]
(defclass test-slot ()
  ((a :initarg :a :index-type slot-index
      :reader test-slot-a
      :index-reader test-slot-with-a
      :index-values all-test-slots)
   (b :initarg :b :index-type slot-index
      :index-reader test-slot-with-b
      :index-subclasses nil
      :index-values all-test-slots-bs))
  (:metaclass indexed-class))

(defclass test-slot2 (test-slot)
  ((b :initarg :b :index-type slot-index
      :index-reader test-slot2-with-b
      :index-subclasses nil
      :index-mapvalues map-test-slot2s
      :index-values all-test-slot2s-bs))
  (:metaclass indexed-class))

(defmethod print-object ((object test-slot) stream)
  (print-unreadable-object (object stream :type t)
    (format stream "~S" (test-slot-a object))))

(make-instance 'test-slot :a 1 :b 2)
(make-instance 'test-slot :a 2 :b 3)
(make-instance 'test-slot2 :a 3 :b 4)
(make-instance 'test-slot2 :a 4 :b 2)
(make-instance 'test-slot2 :a 5 :b 9)

(all-test-slots)
; => (#<TEST-SLOT 1> #<TEST-SLOT 2> #<TEST-SLOT2 3>
;     #<TEST-SLOT2 4> #<TEST-SLOT2 5>)
(test-slot-with-a 2)
; => #<TEST-SLOT 2>
(all-test-slots-bs)
; => (#<TEST-SLOT 1> #<TEST-SLOT 2>)
(all-test-slot2s-bs)
; (#<TEST-SLOT2 3> #<TEST-SLOT2 4> #<TEST-SLOT2 5>)
(map-test-slot2s #'(lambda (obj) (print obj)))
; 
; #<TEST-SLOT2 3> 
; #<TEST-SLOT2 4> 
; #<TEST-SLOT2 5> 
; 
; NIL
\end{Verbatim}
Here is an example of a slot index using an already existing index.

\begin{Verbatim}[fontsize=\small,frame=leftline,framerule=0.9mm,rulecolor=\color{gray},framesep=5.1mm,xleftmargin=5mm,fontfamily=cmtt]
(defvar *existing-slot-index*
  (index-create 'slot-index :slots '(a)))

(defclass test-slot3 ()
  ((a :initarg :a :index *existing-slot-index*))
  (:metaclass indexed-class))

(make-instance 'test-slot3 :a 3)
(make-instance 'test-slot3 :a 4)

(index-get *existing-slot-index* 4)
; => #<TEST-SLOT3 {493B9655}>
; T
(index-values *existing-slot-index*)
; => (#<TEST-SLOT3 {493A0CBD}> #<TEST-SLOT3 {493B9655}>)
\end{Verbatim}
The slot indices of a class can be examined using
`CLASS-SLOT-INDICES'.

\begin{Verbatim}[fontsize=\small,frame=leftline,framerule=0.9mm,rulecolor=\color{gray},framesep=5.1mm,xleftmargin=5mm,fontfamily=cmtt]
(class-slot-indices (find-class 'test-slot) 'a)
; => (#<SLOT-INDEX SLOT: A SIZE: 5 {599FA9F5}>)
(class-slot-indices (find-class 'test-slot) 'b)
; => (#<SLOT-INDEX SLOT: B SIZE: 2 {59A038BD}>)
(class-slot-indices (find-class 'test-slot2) 'a)
; => (#<SLOT-INDEX SLOT: A SIZE: 5 {599FA9F5}>)
(class-slot-indices (find-class 'test-slot2) 'b)
; => (#<SLOT-INDEX SLOT: B SIZE: 3 {59A0D6A5}>)
\end{Verbatim}
Note that a slot can have multiple indices.

\begin{Verbatim}[fontsize=\small,frame=leftline,framerule=0.9mm,rulecolor=\color{gray},framesep=5.1mm,xleftmargin=5mm,fontfamily=cmtt]
(defclass test-slot4 (test-slot)
  ((a :initarg :a :index-type slot-index
      :index-reader test-slot4-with-a
      :index-values all-test-slot4s))
  (:metaclass indexed-class))

(make-instance 'test-slot4 :a 6 :b 9)

(all-test-slots)
; => (#<TEST-SLOT 1> #<TEST-SLOT 2> #<TEST-SLOT2 3>
;     #<TEST-SLOT2 4> #<TEST-SLOT2 5>
;     #<TEST-SLOT4 6>)
(all-test-slot4s)
; => (#<TEST-SLOT4 6>)
(class-slot-indices (find-class 'test-slot4) 'a)
; => (#<SLOT-INDEX SLOT: A SIZE: 6 {599FA9F5}>
;  #<SLOT-INDEX SLOT: A SIZE: 1 {59079E25}>)
\end{Verbatim}


\subsection{ Class indices}

In addition to slot indices, an indexed class supports class
indices which react when one of several slots is changing. For
example, in the `GORILLA' class above, the `COORDS' index reacts
on slots `X' and `Y'. By default, a class index reacts on all
slots.

A class index is created by adding a class option `CLASS-INDICES'
followed by a list of class index specifications.

\begin{Verbatim}[fontsize=\small,frame=leftline,framerule=0.9mm,rulecolor=\color{gray},framesep=5.1mm,xleftmargin=5mm,fontfamily=cmtt]
(defclass test-class ()
  ((x :initarg :x :reader test-class-x)
   (y :initarg :y :reader test-class-y)
   (z :initarg :z :reader test-class-z))
  (:metaclass indexed-class)
  (:class-indices (2d-coords :index-type array-index :slots (x y)
              :index-initargs (:dimensions '(256 256))
              :index-reader test-with-2d-coords)
        (3d-coords :index-type array-index :slots (x y z)
              :index-reader test-with-3d-coords
              :index-initargs (:dimensions '(256 256 2)))))

(defmethod print-object ((object test-class) stream)
  (print-unreadable-object (object stream :type t)
    (with-slots (x y z) object
      (format stream "~d,~d,~d" x y z))))

(make-instance 'test-class :x 1 :y 1 :z 0)
(make-instance 'test-class :x 1 :y 3 :z 1)
(make-instance 'test-class :x 1 :y 2 :z 0)

(test-with-3d-coords '(1 1 0))
; => #<TEST-CLASS 1,1,0>
(test-with-2d-coords '(1 1))
; => #<TEST-CLASS 1,1,0>
(test-with-2d-coords '(1 2))
; => #<TEST-CLASS 1,2,0>
\end{Verbatim}
A class index specification has to comply with the following
lambda-list `(NAME \&REST ARGS \&KEY INDEX-READER INDEX-VALUES SLOTS
TYPE INDEX \&ALLOW-OTHER-KEYS)'. The key arguments `:INDEX-TYPE',
`:INDEX', `:INDEX-READER' and `:INDEX-VALUES' are then removed from
the initargs, and the rest is passed to `INDEX-CREATE' to create
the class index.

`:INDEX-TYPE' - specifies the type of the class index.

`:INDEX' - (optional) specifies an already existing index object
to use.

`:INDEX-READER' - Like `:INDEX-READER' for slot
indices.

`:INDEX-VALUES' - Like `:INDEX-VALUES' for slot indices.

Using `:INDEX', we can use already existing indices as class
indices.

\begin{Verbatim}[fontsize=\small,frame=leftline,framerule=0.9mm,rulecolor=\color{gray},framesep=5.1mm,xleftmargin=5mm,fontfamily=cmtt]
(defvar *array-index*
  (index-create 'array-index :slots '(x y z)
      :dimensions '(256 256 2)))

(defclass test-class2 (test-class)
  ()
  (:metaclass indexed-class)
  (:class-indices (coords :index *array-index* :slots (x y z)
           :index-reader test-with-coords)))

(make-instance 'test-class2 :x 5 :y 5 :z 0)

*array-index*
; => #<ARRAY-INDEX SLOTS: (X Y Z) ((256 256 2)) {593F383D}>
(index-get *array-index* '( 5 5 0))
; => #<TEST-CLASS2 5,5,0>
(test-with-coords '(5 5 0))
; => #<TEST-CLASS2 5,5,0>
\end{Verbatim}
Another example of a class index is the `CLASS-INDEX' index.

\begin{Verbatim}[fontsize=\small,frame=leftline,framerule=0.9mm,rulecolor=\color{gray},framesep=5.1mm,xleftmargin=5mm,fontfamily=cmtt]
(defvar *class-index*
  (index-create 'class-index :index-subclasses t :slot-name 'id))
(defvar *object-id* 0)

(defclass base-object ()
  ((id :initform (incf *object-id*)))
  (:metaclass indexed-class)
  (:class-indices (class :index *class-index*
          :index-reader objects-of-class
          :index-values all-objects
          :index-subclasses t
          :index-keys all-class-names)
        (classes :index-type class-index
            :index-initargs (:index-superclasses t)
            :slots (id)
            :index-subclasses t
            :index-reader objects-with-class)))

(defclass child1 (base-object)
  ()
  (:metaclass indexed-class))

(defclass child2 (base-object)
  ((a :initarg :a))
  (:metaclass indexed-class))

(make-instance 'child1)
(make-instance 'child1)
(make-instance 'child1)
(make-instance 'child2)
(make-instance 'child2)

(all-objects)
; => (#<CHILD1 {48E5CB3D}> #<CHILD1 {48E51395}> #<CHILD1 {48E453DD}>
;  #<CHILD2 {48E82F55}> #<CHILD2 {48E7746D}>)
(objects-with-class 'child1)
; => (#<CHILD1 {48E5CB3D}> #<CHILD1 {48E51395}> #<CHILD1 {48E453DD}>)
; T
(objects-with-class 'child2)
; => (#<CHILD2 {48E82F55}> #<CHILD2 {48E7746D}>)
; T
(objects-with-class 'base-object)
; => (#<CHILD2 {48E82F55}> #<CHILD2 {48E7746D}> #<CHILD1 {48E5CB3D}>
;  #<CHILD1 {48E51395}> #<CHILD1 {48E453DD}>)
; T
(objects-of-class 'child1)
; => (#<CHILD1 {48E5CB3D}> #<CHILD1 {48E51395}> #<CHILD1 {48E453DD}>)
; T
(objects-of-class 'child2)
; => (#<CHILD2 {48E82F55}> #<CHILD2 {48E7746D}>)
; T
(objects-of-class 'base-object)
; => NIL
; NIL
\end{Verbatim}


\subsection{ Destroying objects}

Indexed objects will not be garbage collected until they are
removed from the indices. This is done by calling the
`DESTROY-OBJECT' method on the object. This removes the object
from all its indices, and sets the slot `DESTROYED-P' to `T', so
that not slot-access is possible anymore on the object.

\begin{Verbatim}[fontsize=\small,frame=leftline,framerule=0.9mm,rulecolor=\color{gray},framesep=5.1mm,xleftmargin=5mm,fontfamily=cmtt]
(let ((obj (test-with-coords '(5 5 0))))
       (destroy-object obj))
\end{Verbatim}
This will throw an error.


\begin{Verbatim}[fontsize=\small,frame=leftline,framerule=0.9mm,rulecolor=\color{gray},framesep=5.1mm,xleftmargin=5mm,fontfamily=cmtt]
(test-class-x obj)
\end{Verbatim}


\subsection{ Class and object reinitialization}

When a class is redefined, the indexed-class code tries to map
the new slot-indices to the old-indices. If it finds a slot-index
in the old `EFFECTIVE-SLOT-DEFINITION' and a slot-index in the new
`EFFECTIVE-SLOT-DEFINITION', it calls `INDEX-REINITIALIZE' on the
two indices to copy the values form the old index to the new
one. Afterwards, the same is done for the class
indices. `INDEX-REINITIALIZE' will not be called with the
old-index being the same as the new-index, so that explicitly
instantiated class indices don't get reinitialized with
themselves.

Indices for new slots or new class indices are obviously empty on
creation, and will be filled when the existing instances are
updated. For now, `SHARED-INITIALIZE' is not overloaded, so the
instance update are noticed through `(SETF SLOT-VALUE-USING-CLASS)'.


\section{ Creating a custom index}

The main reason to write indexed slots was to be able to use
custom indices that are appropriate for the task at hand. Indices
are CLOS objects that follow the index method protocol. The
methods that have to be implemented are:

`INDEX-ADD (INDEX OBJECT)' - Add OBJECT to the INDEX. Throws an
ERROR if a problem happened while inserting OBJECT."

`INDEX-GET (INDEX KEY)' - Get the object (or the objects) stored
under the index-key KEY.

`INDEX-REMOVE (INDEX OBJECT)' - Remove OBJECT from the INDEX.

`INDEX-KEYS (INDEX)' - Returns all the keys of the index.

`INDEX-VALUES (INDEX)' - Returns all the objects stored in INDEX.

`INDEX-REINITIALIZE (NEW-INDEX OLD-INDEX)' - Called when the
definition of an index is changed.

`INDEX-CLEAR (INDEX)' - Remove all indexed objects from the index.

In addition to this method, there is the function `INDEX-CREATE'
that instantiates an index object, and calls `INDEX-INITIALIZE' on
it.

The best way to see how this methods are used is to have at look
at the basic index `SLOT-INDEX'. A slot index indexes an object
under a key stored in a slot of this object, so a slot index is
initialized using two arguments: the slot-name where the key is
stored, and a test to create the underlying hash-table.

\begin{Verbatim}[fontsize=\small,frame=leftline,framerule=0.9mm,rulecolor=\color{gray},framesep=5.1mm,xleftmargin=5mm,fontfamily=cmtt]
(defclass slot-index ()
  ((hash-table :initarg :hash-table :accessor slot-index-hash-table
          :documentation "The internal hash table used to index
objects.")
   (slot-name :initarg :slot-name :reader slot-index-slot-name
         :documentation "The value of the slot with name
SLOT-NAME is used as a key to the internal hash-table.")
   (index-nil :initarg :index-nil :reader slot-index-index-nil
         :initform nil
         :documentation "If T, NIL is used as a valid slot
 value, else slots with NIL value are treated as unbound slots.")))

(defmethod initialize-instance :after
    ((index slot-index) &key (test #'eql) slots index-nil
     &allow-other-keys)
  (unless (<= (length slots) 1)
    (error "Can not create slot-index with more than one slot."))
  (with-slots (hash-table slot-name) index
    (setf hash-table (make-hash-table :test test)
     slot-name (first slots)
     (slot-value index 'index-nil) index-nil)))
\end{Verbatim}
When a class is redefined, the indices are re-created. However, we
still want our existing objects to be indexed by the new index,
therefore `INDEX-REINITIALIZE' copies the hash-table when the
hash-table test is the same, or else copies all the stored objects
into the new hash-table.

\begin{Verbatim}[fontsize=\small,frame=leftline,framerule=0.9mm,rulecolor=\color{gray},framesep=5.1mm,xleftmargin=5mm,fontfamily=cmtt]
(defmethod index-reinitialize ((new-index slot-index)
                (old-index slot-index))
  "Reinitialize the slot-bound index from the old index by copying the
internal hash-table if the hash-table test is the same, or by
iterating over the values of the old-table and reentering them into
the new hash-table."
  (let ((new-hash (slot-index-hash-table new-index))
   (old-hash (slot-index-hash-table old-index)))
    (if (eql (hash-table-test new-hash)
        (hash-table-test old-hash))
   (setf (slot-index-hash-table new-index)
         old-hash)
   (loop for key being the hash-keys of old-hash
         using (hash-value value)
         do (setf (gethash key new-hash) value)))
    new-index))
\end{Verbatim}
`INDEX-CLEAR' just creates an empty hash-table to replace the
existing hash-table.

\begin{Verbatim}[fontsize=\small,frame=leftline,framerule=0.9mm,rulecolor=\color{gray},framesep=5.1mm,xleftmargin=5mm,fontfamily=cmtt]
(defmethod index-clear ((index slot-index))
  (with-slots (hash-table) index
    (setf hash-table (make-hash-table
            :test (hash-table-test hash-table)))))
\end{Verbatim}
`INDEX-ADD' and `INDEX-REMOVE' both use the slot-name to get the
key value, and use this key to query the underlying
hash-table. When another object is stored under the key, an error
is thrown.

\begin{Verbatim}[fontsize=\small,frame=leftline,framerule=0.9mm,rulecolor=\color{gray},framesep=5.1mm,xleftmargin=5mm,fontfamily=cmtt]
(defmethod index-add ((index slot-index) object)
  "Add an object using the value of the specified slot as key.
When the hash-table entry already contains a value, an error
is thrown."
  (unless (slot-boundp object (slot-index-slot-name index))
    (return-from index-add))
  (let* ((key (slot-value object (slot-index-slot-name index)))
    (hash-table (slot-index-hash-table index)))
    (when (and (not (slot-index-index-nil index))
          (null key))
      (return-from index-add))
    (multiple-value-bind (value presentp)
   (gethash key hash-table)
      (when (and presentp
       (not (eql value object)))
   (error (make-condition 'index-existing-error
                :index index :key key :value value)))
      (setf (gethash key hash-table) object))))

(defmethod index-remove ((index slot-index) object)
  (let ((slot-name (slot-index-slot-name index)))
    (if (slot-boundp object slot-name)
   (remhash (slot-value object slot-name)
       (slot-index-hash-table index))
   (warn "Ignoring request to remove object ~a
with unbound slot ~A."
         object slot-name))))
\end{Verbatim}
The rest of the methods are straightforward.

\begin{Verbatim}[fontsize=\small,frame=leftline,framerule=0.9mm,rulecolor=\color{gray},framesep=5.1mm,xleftmargin=5mm,fontfamily=cmtt]
(defmethod index-get ((index slot-index) key)
  (gethash key (slot-index-hash-table index)))

(defmethod index-keys ((index slot-index))
  (loop for key being the hash-keys
   of (slot-index-hash-table index)
   collect key))

(defmethod index-values ((index slot-index))
  (loop for value being the hash-values
   of (slot-index-hash-table index)
   collect value))
\end{Verbatim}


\section{ Creating an index using multiple slots}

When creating an index using multiple slots, you have to take care
of a few things. It can happen that a slot-value used by the index
is updated, but that the other slots that are needed are
unbound. However, this is not always an error, so a class index
has to check that all the slots it needs are bound. This is the
`INDEX-ADD' method for an array index.

\begin{Verbatim}[fontsize=\small,frame=leftline,framerule=0.9mm,rulecolor=\color{gray},framesep=5.1mm,xleftmargin=5mm,fontfamily=cmtt]
(defmethod index-add ((index array-index) object)
  (let* ((slot-values
     (mapcar #'(lambda (slot-name)
            ;; return when not all slots are set
            ;;
            ;; - 18.10.04 not needed because of
            ;; make-instance around method
            ;;
            ;; - 19.10.04 in fact this is needed because
            ;; when adding a class index, the existing
            ;; instances are not reinitailized using
            ;; make-instnace, so we have to catch this...
            (unless (slot-boundp object slot-name)
                (return-from index-add nil))
              (slot-value object slot-name))
               (array-index-slot-names index)))
    (array (array-index-array index))
    (dimensions (array-dimensions array)))
    (loop for slot-value in slot-values
     for dimension in dimensions
     when (>= slot-value dimension)
     do (error "Could not add ~a to array-index ~a
because the coordinates ~a are out of bound."
          object index slot-values))
    (let ((value (apply #'aref array slot-values)))
      (when (and value
       (not (eql value object)))
   (error (make-condition 'index-existing-error
                :index index :key slot-values
                :value value))))
    (setf (apply #'aref array slot-values)
     object)))
\end{Verbatim}
